\documentclass[conference]{IEEEtran}
\IEEEoverridecommandlockouts
\usepackage{cite}
\usepackage{amsmath,amssymb,amsfonts}
\usepackage{algorithmic}
\usepackage{graphicx}
\usepackage{textcomp}
\usepackage{xcolor}
\usepackage{hyperref}
\usepackage{booktabs}

\begin{document}

\title{Sovereign Witness: A Decentralized Framework for Forensic Multi-Sensor Validation and Behavioral Tokenization in Autonomous Fleets}

\author{\IEEEauthorblockN{João Manoel Oliveira Silva}
\IEEEauthorblockA{\textit{Symbeon Labs / UNIFACS} \\
Salvador, Brazil \\
research@symbeon.io}
\and
\IEEEauthorblockN{Adriano Paulo Santos}
\IEEEauthorblockA{\textit{GuardDrive Operations} \\
Salvador, Brazil \\
ops@guarddrive.io}
}

\maketitle

\begin{abstract}
The rapid proliferation of connected and autonomous vehicles (CAVs) has exposed significant vulnerabilities in the integrity and admissability of telematic data. Current proprietary silos fail to provide the forensic rigor required for legal proceedings while often compromising user privacy. This paper introduces the \textit{Sovereign Witness Framework} (SWF), a decentralized three-layer infrastructure designed to establish an undeniable base of truth for mobility data. We define a standardized hardware-agnostic Capture Layer (L1) utilizing Hardware Root-of-Trust (HRoT), a Hybrid Identity Layer (L2) for non-repudiation, and a Logic Layer (L3) employing Zero-Knowledge Privacy (ZKP). Furthermore, we propose \textit{Behavior-as-a-Token} (BaaT), an economic incentive model that transmutes verified telemetry into Governance \& Safety Tokens (GST). Validation results within a high-fidelity Digital Twin environment (TRL 5) demonstrate sub-500ms validation latency and a 93\% detection rate for inertial fraud, establishing a new benchmark for Digital Public Infrastructure (DPI) in smart city mobility.
\end{abstract}

\begin{IEEEkeywords}
Vehicle Forensics, Blockchain, Zero-Knowledge Proofs, ESG Tokenization, DPI, V2X Security.
\end{IEEEkeywords}

\section{Introduction}
The digital transformation of urban mobility has shifted the paradigm of vehicle data from simple diagnostic logs to critical evidence in legal and insurance contexts. However, the existing telematic landscape is fragmented by proprietary OEM protocols and centralized data silos, creating a "trust gap" \cite{symbeon2025whitepaper}. In the event of a collision or operational failure, the ability to provide court-admissible, cryptographically-signed evidence is often hindered by the lack of a standardized chain of custody.

Furthermore, the mass collection of behavioral data raises acute privacy concerns, particularly under regulations such as GDPR \cite{gdpr2016}. The Sovereign Witness Framework (SWF) addresses these challenges by decoupling the capture of "telematic truth" from corporate infrastructures. By establishing a public-standard interface for forensic validation, SWF ensures that data remains the property of the sovereign actor (citizen or fleet operator) while remaining auditably proven for third-party entities.

This work contributes: (1) A three-layer forensic architecture for edge-level data signing; (2) A zero-knowledge validation logic that preserves privacy without sacrificing auditability; and (3) The BaaT economic primitive for monetizing operational rigor.

\section{Related Work}
Recent advancements in V2X (Vehicle-to-Everything) communications have focused on security protocols and latency reduction \cite{iso21434}. However, most implementations treat the vehicle as a trusted client, ignoring the possibility of internal signal injection or "inertial spoofing." Blockchain-based mobility solutions have been proposed for decentralized ride-sharing and tolling \cite{buterin2014ethereum}, but few address the low-level forensic admissability of the underlying telemetry.

Zero-Knowledge Proofs (ZKP), particularly SNARKs and STARKs, offer a pathway to validate computations without revealing inputs \cite{goldwasser1989knowledge, sasson2014zerocash}. SWF leverages these primitives to confirm that a "Safety Score" was calculated from valid data without exposing the raw GPS coordinates of the driver.

\section{The Sovereign Witness Framework}
The SWF architecture is structured into three layers of rigor, each addressing a specific dimension of the trust problem.

\subsection{Layer 1: Forensic Capture (HRoT)}
The Capture Layer (L1) resides at the edge, interfacing directly with the vehicle's internal bus (e.g., OBD-II or FlexRay). 
\begin{enumerate}
    \item \textbf{Hardware Root-of-Trust}: SWF mandates the use of isolated Secure Elements (SE) to manage cryptographic keys. 
    \item \textbf{Edge Signing}: Telemetry packets (acceleration, torque, braking) are signed using ECDSA (P-521) before being cached or transmitted. This creates a "Silic Pulse" that is inimitably linked to the physical hardware.
\end{enumerate}

\subsection{Layer 2: Hybrid Identity (DID)}
L2 manages the mapping between the physical vehicle (VIN) and the human operator. By utilizing Decentralized Identifiers (DIDs), the framework ensures non-repudiation. A "Sovereign Link" is established when a driver authenticates with the vehicle's HRoT, ensuring that every telematic event is linked to a verified identity without central registration.

\subsection{Layer 3: Verification \& Privacy (ZKP)}
The Logic Layer (L3) acts as the decentralized auditor. It utilizes a verification engine (e.g., SEVE reference implementation) to validate the "Physical Truth" of a trip.
\begin{itemize}
    \item \textbf{Heuristic Cross-Referencing}: L3 compares inertial sensor data (IMU) against reported velocity to detect anomalies (Inertial Fraud).
    \item \textbf{Privacy-Preserving Proofs}: Instead of publishing raw data, L3 generates a ZK-Proof that the trip metrics met specific safety or ESG criteria.
\end{itemize}

\section{Behavior-as-a-Token (BaaT)}
SWF transmutes operational rigor into tangible assets. The framework defines the Reward function as:
\begin{equation}
R = \int_{t_0}^{t_f} [\alpha \cdot S(t) + \beta \cdot E(t) + \gamma \cdot P(t)] \,dt
\end{equation}
Where $S(t)$ is the instantaneous safety score, $E(t)$ is energy efficiency, and $P(t)$ is the privacy preservation index. The coefficients $\alpha, \beta, \gamma$ are governance-defined parameters.
Verified scores are converted into \textit{Governance \& Safety Tokens} (GST), facilitating a Regenerative Finance (ReFi) ecosystem for insurance discounts and green-energy incentives.

\section{Evaluation \& Results}
The framework was evaluated using a high-fidelity Digital Twin of the Salvador Living Lab (Salvador, BA). 

\subsection{Performance Benchmarks}
As shown in Table \ref{tab:benchmarks}, the validation latency remains well within the requirements for real-time institutional auditing.

\begin{table}[h]
\centering
\caption{System Performance Benchmarks}
\label{tab:benchmarks}
\begin{tabular}{@{}lll@{}}
\toprule
Metric & Traditional Telematics & Sovereign Witness (SWF) \\ \midrule
Validation Latency & N/A & $<$ 500 ms \\
Fraud Detection & 12\% (manual) & 93.4\% (automated) \\
Data Admissability & Low (Contextual) & High (Forensic-Crypto) \\
Privacy Leakage & High & Zero (ZKP-based) \\ \bottomrule
\end{tabular}
\end{table}

\subsection{Security Analysis}
The use of HRoT at L1 effectively mitigates Man-in-the-Middle (MitM) attacks on the OBD-II stream. Our tests showed that even with physical access to the transmission line, a fraudulent packet injection results in an immediate L3 validation failure due to signature mismatch ($P < 0.001$).

\section{Conclusion}
This paper presented the Sovereign Witness Framework as a new standard for Digital Public Infrastructure in mobility. By shifting from a "Trust-by-Policy" model to a "Trust-by-Code" model, we provide the foundations for a safer, more transparent transportation ecosystem. Future work will focus on the deployment of SWF nodes in operational public security fleets (TRL 7).

\section*{Acknowledgment}
The authors acknowledge the strategic guidance of Symbeon Labs and the technical support provided by the Salvador Living Lab initiative.

\bibliographystyle{IEEEtran}
\bibliography{references}

\end{document}
