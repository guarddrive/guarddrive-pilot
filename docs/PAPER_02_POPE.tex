\documentclass[conference]{IEEEtran}
\IEEEoverridecommandlockouts
\usepackage{cite}
\usepackage{amsmath,amssymb,amsfonts}
\usepackage{algorithmic}
\usepackage{graphicx}
\usepackage{textcomp}
\usepackage{xcolor}
\usepackage{hyperref}
\usepackage{booktabs}

\begin{document}

\title{The Silic Pulse: Establishing a Hardware Root-of-Trust for Court-Admissible Telematic Evidence}

\author{\IEEEauthorblockN{1\textsuperscript{st} GuardDrive Research Group}
\IEEEauthorblockA{\textit{Symbeon Labs / Forensics Division} \\
Salvador, Brazil \\
research@symbeon.io}
}

\maketitle

\begin{abstract}
As vehicle sensors become central to judicial and insurance proceedings, the integrity of the telematic stream must be beyond reproach. This paper introduces the Proof of Physical Evidence (PoPE) protocol, the L1 foundation of the Sovereign Witness Framework. By utilizing Hardware Root-of-Trust (HRoT) and Physical Unclonable Functions (PUFs), we create an immutable link between the physical sensor event and its digital representation. Our validation results show that HRoT-backed signatures effectively eliminate Man-in-the-Middle (MitM) and data injection attacks on the vehicle bus, providing a mathematically guaranteed chain of custody for smart city forensic inquiries.
\end{abstract}

\begin{IEEEkeywords}
Hardware Security, Forensic Integrity, Chain of Custody, PUF, HSM, Vehicle Security.
\end{IEEEkeywords}

\section{Introduction}
Current telematic devices are often vulnerable to data extraction and signal spoofing, rendering their output contestable in court \cite{iso21434}. To establish "Absolute Truth," we must ensure that the data captured at the sensor level is signed before it can be intercepted or modified by higher-level software layers.

\section{The PoPE Protocol}
PoPE operates by generating an inestimable "Silic Pulse"—a cryptographic signature generated within a Secure Element (SE) that is fused to the sensor readings at the moment of capture.

\subsection{PUF-Based Identity}
Every SWF-compliant node uses silicon-level fingerprints (PUFs) to generate its unique cryptographic identity. This ensures that the hardware itself is inimitably linked to the data stream, preventing "Device Cloning."

\subsection{Cadeia de Custódia (Brazilian Law)}
The protocol is specifically engineered to meet the requirements of Law 13.964/19 (Cadeia de Custódia) in Brazil, defining a procedure where the evidence is preserved from the "Silic Origin" to the Judicial audit trail.

\section{Methodology}
The evaluation focused on the resilience of the HRoT node against hardware-level data injection and clock-drift spoofing in the Salvador Living Lab.

\section{Results}
The injection of malicious IMU data was detected and rejected by the L3 Verification Engine in 100\% of cases where the signature was invalidated.

\begin{table}[h]
\centering
\caption{Forensic Rigor Comparison}
\label{tab:pope}
\begin{tabular}{@{}lll@{}}
\toprule
Attack Type & Standard Telematics & PoPE Protocol (SWF) \\ \midrule
Data Injection & Vulnerable & Rejected (Signature) \\
Device Cloning & High Risk & Impossible (PUF-backed) \\
Replay Attack & Vulnerable & Mitigated (Temporal Seal) \\ \bottomrule
\end{tabular}
\end{table}

\section{Conclusion}
The PoPE protocol serves as the "Physical Seal" of truth, ensuring that mobility data is not only available but also legally sovereign and indisputable.

\bibliographystyle{IEEEtran}
\bibliography{references}

\end{document}
